The chance of the patient becoming functional when starting out in state 1 is given by the following expression:

\begin{equation}
    P(\text{Patient\ becomes\ functional}) = (\sum_{n=1}^{\infty}P_{03}{\cdot}P_{10}^{n} + P_{13}{\cdot}P_{11}^{n} + P_{23}{\cdot}P_{12}^{n}  + P_{33}{\cdot}P_{13}^{n}) + P_{13}
\end{equation}

Since $P_{03}$, $P_{13}$ and $P_{33}$ are all equal to zero,  we get

\begin{equation}
    P_{23}{\cdot}\sum_{n=1}^{\infty}P_{12}^{n} = 0.3 \cdot \sum_{n=1}^{\infty}0.1^{n} = 0.3 \cdot \frac{0.1}{1 - 0.1} = \frac{3}{10} \cdot \frac{1}{9} = \frac{1}{30}
\end{equation}

The transition matrix can be reorganized so that it is of the following form

\begin{figure}[htbp]
    \label{P}
    \centering
    $\mathbf{P} = 
            \begin{blockarray}{c@{\hspace{1pt}}rrrr@{\hspace{3pt}}}
            \begin{block}{r@{\hspace{1pt}}|@{\hspace{1pt}}
        |@{\hspace{1pt}}rr@{\hspace{1pt}}|@{\hspace{1pt}}|}
            & Q &  R \\
            & 0 & I  \\
            \end{block}
        \end{blockarray}$
\end{figure}

with Q representing the matrix that describes steps between transient states and R representing the matrix that describes steps between transient and absorbing states. With at least one absorbing state we also get the identity matrix $I$ and a null matrix. 

We want to find the expected amount of steps before we enter one of the two absorbing steps. We approach this problem by looking at the fundamental matrix N given by


\begin{equation}
    N = \sum_{n=1}^{\infty}Q^{n} = (I - Q)^{-1}
\end{equation}

We get the matrix 

\begin{figure}[htbp]
    \label{P}
    \centering
    $\mathbf{P} = 
            \begin{blockarray}{c@{\hspace{1pt}}rrrr@{\hspace{3pt}}}
             & 1   & 2   & 0   & 3   \\
            \begin{block}{r@{\hspace{1pt}}|@{\hspace{1pt}}
        |@{\hspace{1pt}}rrrr@{\hspace{1pt}}|@{\hspace{1pt}}|}
            1 & 0.85 & 0.1 & 0.05 & 0 \\
            2 & 0.05 & 0.65 & 0 & 0.3 \\
            0 & 0   & 0   & 1 & 0   \\
            3 & 0 & 0 & 0 & 1 \\
            \end{block}
        \end{blockarray}$
\end{figure}

with 

\begin{figure}[htbp]
    \label{Q}
    \centering
    $\mathbf{Q} = 
            \begin{blockarray}{c@{\hspace{1pt}}rrrr@{\hspace{3pt}}}
            \begin{block}{r@{\hspace{1pt}}|@{\hspace{1pt}}
        |@{\hspace{1pt}}rr@{\hspace{1pt}}|@{\hspace{1pt}}|}
            & 0.85 &  0.1 \\
            & 0.05 & 0.65  \\
            \end{block}
        \end{blockarray}$
\end{figure}

the expected number of steps is given by the first element in $t = N1$ with N being the fundamental matrix and 1 being a column vector of size t and entries all 1. 


<<>>=
Q <- matrix(data=c(0.85, 0.05,0.1,0.65), nrow = 2)
I <- diag(2)
N <- solve(I-Q)
t <- N%*%matrix(data=c(1,1),ncol=1)
t[1]
@
